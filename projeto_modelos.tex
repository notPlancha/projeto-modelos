\documentclass[a4paper, justified]{tufte-handout}\usepackage[]{graphicx}\usepackage[]{xcolor}
% maxwidth is the original width if it is less than linewidth
% otherwise use linewidth (to make sure the graphics do not exceed the margin)
\makeatletter
\def\maxwidth{ %
  \ifdim\Gin@nat@width>\linewidth
    \linewidth
  \else
    \Gin@nat@width
  \fi
}
\makeatother

\definecolor{fgcolor}{rgb}{0.345, 0.345, 0.345}
\newcommand{\hlnum}[1]{\textcolor[rgb]{0.686,0.059,0.569}{#1}}%
\newcommand{\hlstr}[1]{\textcolor[rgb]{0.192,0.494,0.8}{#1}}%
\newcommand{\hlcom}[1]{\textcolor[rgb]{0.678,0.584,0.686}{\textit{#1}}}%
\newcommand{\hlopt}[1]{\textcolor[rgb]{0,0,0}{#1}}%
\newcommand{\hlstd}[1]{\textcolor[rgb]{0.345,0.345,0.345}{#1}}%
\newcommand{\hlkwa}[1]{\textcolor[rgb]{0.161,0.373,0.58}{\textbf{#1}}}%
\newcommand{\hlkwb}[1]{\textcolor[rgb]{0.69,0.353,0.396}{#1}}%
\newcommand{\hlkwc}[1]{\textcolor[rgb]{0.333,0.667,0.333}{#1}}%
\newcommand{\hlkwd}[1]{\textcolor[rgb]{0.737,0.353,0.396}{\textbf{#1}}}%
\let\hlipl\hlkwb

\usepackage{framed}
\makeatletter
\newenvironment{kframe}{%
 \def\at@end@of@kframe{}%
 \ifinner\ifhmode%
  \def\at@end@of@kframe{\end{minipage}}%
  \begin{minipage}{\columnwidth}%
 \fi\fi%
 \def\FrameCommand##1{\hskip\@totalleftmargin \hskip-\fboxsep
 \colorbox{shadecolor}{##1}\hskip-\fboxsep
     % There is no \\@totalrightmargin, so:
     \hskip-\linewidth \hskip-\@totalleftmargin \hskip\columnwidth}%
 \MakeFramed {\advance\hsize-\width
   \@totalleftmargin\z@ \linewidth\hsize
   \@setminipage}}%
 {\par\unskip\endMakeFramed%
 \at@end@of@kframe}
\makeatother

\definecolor{shadecolor}{rgb}{.97, .97, .97}
\definecolor{messagecolor}{rgb}{0, 0, 0}
\definecolor{warningcolor}{rgb}{1, 0, 1}
\definecolor{errorcolor}{rgb}{1, 0, 0}
\newenvironment{knitrout}{}{} % an empty environment to be redefined in TeX

\usepackage{alltt}
\geometry{
  left=24.8mm, % left margin
  textwidth=100mm, % main text block
  marginparsep=3mm, % gutter between main text block and margin notes
  marginparwidth=75mm % width of margin notes
}
% Packages {
  \usepackage[portuguese]{babel}
  % KableExtra {
    \usepackage{booktabs}
    \usepackage{longtable}
    \usepackage{array}
    \usepackage{multirow}
    \usepackage{wrapfig}
    \usepackage{float}
    \usepackage{colortbl}
    \usepackage{pdflscape}
    \usepackage{tabu}
    \usepackage{threeparttable}
    \usepackage{threeparttablex}
    \usepackage[normalem]{ulem}
    \usepackage{makecell}
    \usepackage{xcolor}
    \usepackage{comment}
    \usepackage{lipsum}
    \usepackage{tabulary}
    \usepackage{tabularx}
    %\usepackage{showframe}
  % }
% }



% Titulo {
  \title{TODO TITULO}
  \author{
    André Plancha, 105289 \\
    <Andre\_Plancha@iscte-iul.pt> \\
    Tomás Ribeiro, 105220 \\
    <tfroo1@iscte-iul.pt> \\
    Afonso Silva, 105208 \\
    <agsos@iscte-iul.pt> \\
    Rui Chaves, 104914 \\
    rfpcs1@iscte-iul.pt
  }
  \date{27/11/2022 \\ Versão 0.0.1} % 
% }

% { bug workaround (https://github.com/Tufte-LaTeX/tufte-latex/issues/64#issuecomment-78572017)
\renewcommand\allcapsspacing[1]{{\addfontfeature{LetterSpace=15}#1}}
\renewcommand\smallcapsspacing[1]{{\addfontfeature{LetterSpace=10}#1}}
% }
% {R code and table
\newenvironment{RTable}[1]{\begin{margintable}\begin{center}\begin{tabular}{#1}}{\end{tabular}\end{center}\end{margintable}}
\newenvironment{RTable*}[1]{\begin{table*}\begin{center}\begin{tabular}{#1}}{\end{tabular}\end{center}\end{table*}}
% }
\IfFileExists{upquote.sty}{\usepackage{upquote}}{}
\begin{document}

\begin{knitrout}
\definecolor{shadecolor}{rgb}{0.969, 0.969, 0.969}\color{fgcolor}\begin{kframe}
\begin{verbatim}
## standardGeneric for "show" defined from package "methods"
## 
## function (object) 
## standardGeneric("show")
## <bytecode: 0x00000198cb4a6578>
## <environment: 0x00000198ca122220>
## Methods may be defined for arguments: object
## Use  showMethods(show)  for currently available ones.
## (This generic function excludes non-simple inheritance; see ?setIs)
\end{verbatim}
\end{kframe}
\end{knitrout}

% cover { 
  \cleardoublepage
  {
  \sffamily
  \begin{fullwidth}
  \fontsize{18}{20}\selectfont\par\noindent\textcolor{darkgray}{\allcaps{\thanklessauthor}}%
  \vspace{11.5pc}
  \fontsize{36}{40}\selectfont\par\noindent\textcolor{darkgray}{\allcaps{\thanklesstitle}}%
  \vfill
  \fontsize{14}{16}\selectfont\par\noindent\allcaps{\thanklesspublisher}
  \end{fullwidth}
  }
  \thispagestyle{empty}
  \clearpage
% }
% abstract
  \newgeometry{left=30mm, right=107mm, textwidth=100mm}
  \vspace*{1cm}
  \begin{fullwidth}
    \Large{
      Hello 
      \lipsum[2]
      \vspace{5cm}
      \lipsum[3]
      \thispagestyle{empty}
    }
    \clearpage
  \end{fullwidth}
  \restoregeometry
% }
% descricao airbnb TODO
% also o airbnb n vende quartos
O nosso trabalho tem como objetivo desenvolver um modelo que permita prever o preço a que os quartos são colocados á venda no AIrbnb, com vista nesse objetivo foi nos disponibilizado uma base de dados que têm várias informações úteis para realizar esta tarefa.
Usámos como suporte técnico a linguagem R através do uso de um computador pessoal quer para a limpeza de dados, o tratamento de dados, e as respetivas previsões efectuadas.

\begin{knitrout}
\definecolor{shadecolor}{rgb}{0.969, 0.969, 0.969}\color{fgcolor}\begin{kframe}
\begin{alltt}
\hlstd{listings} \hlkwb{<-} \hlkwd{read.csv}\hlstd{(}\hlkwd{here}\hlstd{(}\hlstr{"data"}\hlstd{,} \hlstr{"listings.csv"}\hlstd{))}
\hlkwd{nrow}\hlstd{(listings)}
\end{alltt}
\begin{verbatim}
## [1] 6629
\end{verbatim}
\end{kframe}
\end{knitrout}

A base de dados que nos foi disponibilizada vem do projeto \href{http://insideairbnb.com/}{Inside Airbnb}, fundado por Murray Cox com a missão de "[...] fornecer dados e defesa sobre o impacto do Airbnb em comunidades residenciais"\cite{InsideAirBnbAbt}.

A base de dados contém 6629 entradas, e cada uma delas representa um registo de um anúncio para o aluguer de um alojamento disponível no Airbnb, em São Francisco, Califórnia. Cada alojamento contém informação sobre o seu preço, localização, hospedeiro, o tipo de alojamento, as \textit{reviews} do alojamento, e licensa do alojamento.

\begin{knitrout}
\definecolor{shadecolor}{rgb}{0.969, 0.969, 0.969}\color{fgcolor}\begin{kframe}
\begin{alltt}
\hlkwd{data.frame}\hlstd{(}\hlkwc{row.names} \hlstd{=} \hlkwd{colnames}\hlstd{(listings),}
    \hlkwc{type} \hlstd{=} \hlkwd{sapply}\hlstd{(listings, class))} \hlopt
    \hlkwd{showT}\hlstd{()}
\end{alltt}
\end{kframe}\begingroup\fontsize{10}{12}\selectfont

\begin{RTable}{l|c}
\hline
  & type\\
\hline
\cellcolor{gray!6}{id} & \cellcolor{gray!6}{numeric}\\
\hline
name & character\\
\hline
\cellcolor{gray!6}{host\_id} & \cellcolor{gray!6}{integer}\\
\hline
host\_name & character\\
\hline
\cellcolor{gray!6}{neighbourhood\_group} & \cellcolor{gray!6}{logical}\\
\hline
neighbourhood & character\\
\hline
\cellcolor{gray!6}{latitude} & \cellcolor{gray!6}{numeric}\\
\hline
longitude & numeric\\
\hline
\cellcolor{gray!6}{room\_type} & \cellcolor{gray!6}{character}\\
\hline
price & integer\\
\hline
\cellcolor{gray!6}{minimum\_nights} & \cellcolor{gray!6}{integer}\\
\hline
number\_of\_reviews & integer\\
\hline
\cellcolor{gray!6}{last\_review} & \cellcolor{gray!6}{character}\\
\hline
reviews\_per\_month & numeric\\
\hline
\cellcolor{gray!6}{calculated\_host\_listings\_count} & \cellcolor{gray!6}{integer}\\
\hline
availability\_365 & integer\\
\hline
\cellcolor{gray!6}{number\_of\_reviews\_ltm} & \cellcolor{gray!6}{integer}\\
\hline
license & character\\
\hline
\end{RTable}
\endgroup{}

\end{knitrout}
De forma a perceber melhor a base dados, o \textit{Airbnb} disponibiliza de um "dicionário de dados"\cite{DicDadosAirBnb} que explica o significado de cada uma das variáveis:

\begin{itemize}
  \item \textbf{id}: Número que representa um identificador único do anúncio;
  \item \textbf{name}: Título do anúncio;
  \item \textbf{host\_id}: Identificador único da conta do hospedeiro;
  \item \textbf{host\_name}: Nome da conta do hospedeiro \sidenote{Normalmente este campo inclui apenas o primeiro nome ou nome da instituição hospedeira};
  \item \textbf{neighbourhood\_group}: Este campo encontra-se vazio e não inclui descrição no dicionário;
  \item \textbf{neighbourhood}: Embora este campo não inclua descrição no dicionário, nesta base de dados este campo descreve bairros de São Francisco como definido pelo Departamento de Planejamento da cidade \sidenote{Os bairros de São Francisco não contém fronteiras oficiais e dependem da fonte (\url{tldrify.com/19p8}), logo a definição das fronteiras definidas pelo Airbnb tiveram de ser determinadas. Mais à frente será demonstrado as fronteiras};
  \item \textbf{latitude/longitude}: Coordenadas geográficas do alojamento;
  \item \textbf{room\_type}: Tipo de alojamento, entre "Quarto privado", "Quarto partilhado", "Quarto de hotel", e "Casa/Apartamento inteiro";
  \item \textbf{price}: Preço do alojamento por noite em USD;
  \item \textbf{minimum\_nights}: Número mínimo de noites que o hospedeiro exige para alugar o alojamento;
  \item \textbf{number\_of\_reviews}: Número total de \textit{reviews} que o alojamento tem;
  \item \textbf{last\_review}: Data da última \textit{review} que o alojamento recebeu;
  \item \textbf{reviews\_per\_month}: Número médio de \textit{reviews} que o alojamento recebe por mês;
  \item \textbf{calculated\_host\_listings\_count}: Número de alojamentos que o hospedeiro tem disponíveis em São Francisco;
  \item \textbf{availability\_365}: Número de dias que o alojamento está disponível por ano.
  \item \textbf{number\_of\_reviews\_ltm}: Número de \textit{reviews} que o alojamento recebeu nos últimos 12 meses;
  \item \textbf{license}: A licença/autorização/número de registo do alojamento.
\end{itemize}

\begin{knitrout}
\definecolor{shadecolor}{rgb}{0.969, 0.969, 0.969}\color{fgcolor}\begin{kframe}
\begin{alltt}
\hlcom{# listings %>% head(1) %>% t() %>%}
\hlcom{# showT(T)}
\end{alltt}
\end{kframe}
\end{knitrout}

\nobibliography{references.bib}
\bibliographystyle{plainnat}
\end{document}
