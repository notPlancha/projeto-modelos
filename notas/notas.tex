\documentclass{assignment}

\usepackage{markdown}

\begin{document}

\section*{Introdução}
A regressão tem por objetivo explicar e prever fenómenos

a formalização de um modelo nbão é mais do que uma aproximaçãoaos aspectos essencias da realidade em estudo

\subsection*{Grandezas}
Dois tipos: parâmetros e variáveis; e dois tipos de variáveis: exógenas/independentes, e endógenas/dependentes: exógenas são determinadas por causas externas ao modelo e são introduzidas para explicar variáveis endógenas; no entanto as independentes não causam as dependentes necessariamente.

\subsection*{Modelos}
Regressão pretende explicar e prever o comportamento da variável Y em função das variáveis X, necessitando de uma expressão analítica que a traduz, obtida através do Método dos Mínimos Quadrados (MMQ), ou \textit{Ordinary Least Squares} (OLS).


Modelo de regressão linear múltipla:
\begin{equation}
Y = \beta_0 + \beta_1 X_1 + \beta_2 X_2 + \cdots + \beta_k X_k + \epsilon
\end{equation}
onde
\begin{itemize}
  \item $Y$ é a variável dependente;
  \item $X_1, X_2, \cdots, X_k$ são as variáveis independentes;
  \item $\beta_0, \beta_1, \beta_2, \cdots, \beta_k$ são os parâmetros do  modelo, constantes;
  \item $\epsilon$ é o erro aleatório, não observável, a qual inclui todas as influências em $Y$ que não são explicados por $X_i$.
\end{itemize}
Os modelos de regressão incluem as seguintes etapas:
\begin{enumerate}
  \item Exploração dos dados;
  \item Verificação dos pressupostos;
  \item Estimação dos parâmetros;
  \item Deteção de outliers;
  \item Previsão;
  \item Critica ao modelo.
\end{enumerate}

\section*{Pressupostos}

\url{https://sphweb.bumc.bu.edu/otlt/MPH-Modules/BS/R/R5_Correlation-Regression/R5_Correlation-Regression4.html}

\begin{enumerate}
  \item Linearidade: a relação entre $X$ e $E[Y]$ É É linear;
  \item Independência: cada observação é independente das outras;
  \item Homocedasticidade: a variância de $E[Y]$ é constante;
  \item Normalidade: $Y$ é normalmente distribuída para qualquer X;
\end{enumerate}








\end{document}
